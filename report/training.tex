\section{Training}
% What went wrong during multiple iterations of training?
This section covers the training regime and a comparison between CPU and GPU training.
The code is available on GitLab\footnote{\url{https://git.hhu.de/nirec101/transformer_project}}.

\subsection{Data}
We train on the WMT 17 German to English dataset\footnote{\url{https://www.statmt.org/wmt17/translation-task.html}}, consisting of about 4.9 million sentence pairs after filtering out sequences exceeding 64 tokens in length.
To not inflict unnecessary load on the GPU during training, we preprocess the datasets berforehand.
The sequences are encoded using byte-pair encoding~\cite{britz2017massiveexplorationneuralmachine}, which has a shared source-target vocabulary of 50000 tokens.
We use only 5\% of the training data for benchmarking CPU vs. GPU performance with a batch size of 32 on both.
For the final model perfomance reported in \cref{sec:results}, however, we train on the entire dataset with a batch size of 1024 sentence pairs.

\subsection{Training and Schedule}
We train our models on a single node with five processing cores and a single NVIDIA A100 GPU for XXX steps (10 epochs).
Since GPUs are optimized for parallel computation, they are well-suited for the highly parallel nature of sequence processing in Transformer models.
Unlike CPUs, which are designed for handling a wide range of sequential operations, GPUs distribute the workload across its many cores, consisting of 8192 FP32 CUDA cores and 432 Tensor cores\footnote{\url{https://images.nvidia.com/aem-dam/en-zz/Solutions/data-center/nvidia-ampere-architecture-whitepaper.pdf}}, optimized for matrix operations.
For benchmarking CPU vs. GPU performance under identical conditions, we use the exact same set of hyperparameters on both setups, with the CPU configuration consiting of five cores and 64GB of memory.
Each step took about XXX seconds, utilizing mixed precision.


\subsection{AdamW Optimizer}
In all the experiments, we use the AdamW optimizer~\cite{loshchilov2019decoupledweightdecayregularization} with \(\beta_1=0.9\), \(\beta_2=0.99\) and \(\epsilon=10^{-8}\).
This section elaborates the core differences between the Adam optimizer~\cite{kingma2017adammethodstochasticoptimization} used in the original Transformer architecture and AdamW.\\
Both, in Adam and AdamW, \cref{eq:mean_var} shows that the learning rate is adjusted for each parameter independently based on the history of gradients.
The running averages, \(m_{t-1}\) and \(v_{t-1}\), make it possible to include the history of the gradients in the calculation of the first and second moment:

% Equation 1: First moment estimate
\begin{equation}
\label{eq:mean_var}
m_t = \beta_1 m_{t-1} + (1 - \beta_1) g_t \text{,} \quad v_t = \beta_2 v_{t-1} + (1 - \beta_2) g_t^2
\end{equation}
The calculation of the first and second moment in this fashion ensures that parameters with larger gradient variances are updated more slowly than those with larger gradient variances to stabilize the optimization process. \\
The bias correction from \cref{eq:moments} is important because, without it, the first and second moments are biased toward zero at early time steps, because \(m_0\) and \(v_0\) are zero. Consequently, this results in overly careful parameter updates in the beginning, which hinder the performance and convergence of the training process. \\
% Equation 3: Bias-corrected moments
\begin{equation}
\hat{m}_t = \frac{m_t}{1 - \beta_1^t} \text{,} \quad \hat{v}_t = \frac{v_t}{1 - \beta_2^t}
\label{eq:moments}
\end{equation}
\\
In the original Adam, weight decay is added directly to the gradient. Consequently, this means that the weight decay term is included in the moment estimates (\(m_t\) and \(v_t\)). The AdamW optimizer circumvents this problem: The weight decay is applied directly to the weights after the adaptive gradient update, as shown in \cref{eq:weight_decay}:
% Equation 5: Weight decay
\begin{equation}
\label{eq:weight_decay}
\theta_t \leftarrow \theta_t - \eta \lambda \theta_t
\end{equation}
\\
\cref{eq:update} shows the complete parameter update for the AdamW optimizer, where the weight decay is decoupled from the gradient calculation.
% Equation 6: Parameter update
\begin{equation}
\label{eq:update}
\theta_{t+1} = \theta_t - \eta \left( \frac{\hat{m}_t}{\sqrt{\hat{v}_t} + \epsilon} \right) - \eta \lambda \theta_t
\end{equation}
\todo[inline]{Add figure from AdamW paper?}




In the backward pass, if you have multiple rank-deficient matrices, your rank becomes even lower.
because the composition of rank-deficient matrices leads to a further reduction in the rank, potentially causing the gradients to vanish or lose critical information needed for effective weight updates.
